% !TeX spellcheck = pl_PL
\documentclass[12pt]{oska}

% Lista wszystkich języków stanowiących języki pozycji bibliograficznych użytych w pracy.
% (Zgodnie z zasadami tworzenia bibliografii każda pozycja powinna zostać utworzona zgodnie z zasadami języka, w którym dana publikacja została napisana.)
\usepackage[english,polish]{babel}

% Użyj polskiego łamania wyrazów (zamiast domyślnego angielskiego).
\usepackage{polski}

\usepackage[utf8]{inputenc}

% dodatkowe pakiety
\usepackage{mathtools}
\usepackage{amsfonts}
\usepackage{amsmath}
\usepackage{amsthm}
\usepackage[dvipsnames]{xcolor}
\usepackage{textcomp}

% obrazki
\usepackage{graphicx}
\usepackage{rotating}
\usepackage{caption}
\usepackage{float}

% --- < bibliografia > ---

\usepackage[
style=numeric,
sorting=none,
% Zastosuj styl wpisu bibliograficznego właściwy językowi publikacji.
language=autobib,
autolang=other,
% Zapisuj datę dostępu do strony WWW w formacie RRRR-MM-DD
%urldate=iso,
%seconds=true,
% Nie dodawaj numerów stron, na których występuje cytowanie
backref=false,
% Podawaj ISBN.
isbn=true,
% Nie podawaj URL-i, o ile nie jest to konieczne
url=false,
% Ustawienia związane z polskimi normami dla bibliografii
maxbibnames=6,
minbibnames=6,
% Jeżeli używamy Bibera:
backend=biber
]{biblatex}

\AtBeginBibliography{
\renewcommand\labelnamepunct{:\space}
\renewcommand\newunitpunct{\addcomma\space}
\renewcommand{\finentrypunct}{}

\renewcommand{\bibopenparen}{\addcomma\addspace}
\renewcommand{\bibcloseparen}{\addspace}
}

\usepackage{csquotes}
% Ponieważ `csquotes` nie posiada polskiego stylu, można skorzystać z mocno zbliżonego stylu chorwackiego.
\DeclareQuoteAlias{croatian}{polish}

% Przecinki do numerów
\usepackage{icomma}
% ------------------------

% --- < listingi > ---

% Użyj czcionki kroju Times.
\usepackage{newtxtext}

\usepackage{listings}
\lstloadlanguages{TeX}

\lstset{
	literate={ą}{{\k{a}}}1
           {ć}{{\'c}}1
           {ę}{{\k{e}}}1
           {ó}{{\'o}}1
           {ń}{{\'n}}1
           {ł}{{\l{}}}1
           {ś}{{\'s}}1
           {ź}{{\'z}}1
           {ż}{{\.z}}1
           {Ą}{{\k{A}}}1
           {Ć}{{\'C}}1
           {Ę}{{\k{E}}}1
           {Ó}{{\'O}}1
           {Ń}{{\'N}}1
           {Ł}{{\L{}}}1
           {Ś}{{\'S}}1
           {Ź}{{\'Z}}1
           {Ż}{{\.Z}}1,
	basicstyle=\footnotesize\ttfamily,
}

% ------------------------

\AtBeginDocument{
	\renewcommand{\tablename}{\textbf{Tabela}}
	\renewcommand{\figurename}{\textbf{Rysunek}}
}

% ------------------------
% --- < tabele > ---

\usepackage{array}
\usepackage{tabularx}
\usepackage{multirow}
\usepackage{booktabs}
\usepackage{makecell}
\usepackage[flushleft]{threeparttable}

% defines the X column to use m (\parbox[c]) instead of p (`parbox[t]`)
\newcolumntype{C}[1]{>{\hsize=#1\hsize\centering\arraybackslash}X}

\setlength{\cftsecnumwidth}{10mm}
\setcounter{secnumdepth}{4}
\brokenpenalty=10000\relax

% ------------------------
% --- < do tego artykułu > ---
\sisetup{binary-units=true}
\DeclareSIUnit{\inch}{"}

%---------------------------------------------------------------------------

\titlePL{Wielokanałowy system syntezy pola akustycznego}
\titleEN{Multichannel acoustic field synthesis system}
\affiliation{Akademia Górniczo-Hutnicza im. Stanisława Staszica w Krakowie}

%------------------------------------AUTORZY-----------------------

\namem{Marcel}
\surnamem{Piszak}
\email{marcel.piszak@gmail.com} % adres do korespondencji -- zazwyczaj głównego autora

% Jeśli jesteś jedynym autorem pracy - pozostaw poniższe pola puste

\namei{Szymon}
\surnamei{Mikulicz}

\nameii{Teresa}
\surnameii{Makuch}

\nameiii{Michał}
\surnameiii{Kmiecik}

\nameiiii{}
\surnameiiii{}

\nameiiiii{}
\surnameiiiii{}

%--------------------------STRESZCZENIE------------------------

\summaryPL{}

\summaryEN{}

%---------------------------------------------------------
% Nazwa pliku z bibliografią
%---------------------------------------------------------
\addbibresource{bibliografia.bib}


\begin{document}

\maketitles

\section{Wstęp}

Reprodukcja dźwięku jest bardzo szerokim zagadnieniem akustyki. W obecnych
czasach istnieje wiele technik pozwalających na odtworzenie panujących w
wybranym miejscu warunków akustycznych w zgoła innym otoczeniu. Złożoność
zjawisk związanych z propagacją fali akustycznej w rzeczywistym środowisku
sprawia, że omawiane zagadnienie staję się złożonym problemem. Jedną z
naistotniejszych trudności napotykanych podczas projektowania systemu
reprodukcji dźwięku jest ograniczenie przestrzeni dokładnego odwzorowania do
punku, a właściwie niewielkiego obszaru wokół niego (tzw. sweet-spot).
Istnienie takiego ograniczenia wyklucza zastosowanie takich systemów w
sytuacjach, gdy mamy do czynienia z odbiornikiem wykonującym ruch w strefie
odsłuchu. Z pomocą przychodzą jednak systemy syntezy pola akustycznego (Wave
Field Synthesis - WFS), które bazując na dużej liczbie głośników są w stanie
imitować fale akustyczne generowane przez źródła usytuowane nawet w dużej
odległości oraz będące w ruchu, ale co najważniejsze nie ograniczają strefy
odsłuchu do punktu. Obecnie istnieją różne warianty systemów WFS, projektowane
często z myślą o konkretnym zastosowaniu [cite cośtam]. Celem niniejszej pracy
było zaprojektowanie modułowego systemu syntezy pola akustycznego, opartego na
łatwo dostępnych komponentach i umożliwiającego stosunkowo prostą adaptację do
wybranego zastosowania.

\section{Teoria syntezy pola akustycznego}

[tutaj trzeba dać sporo o fizyce, wzorach, itp.]

\section{Projekt modułowego systemu WFS}

Zdecydowano się na zastosowanie rozwiązania opartego na komputerach Raspberry
Pi Zero. Są one niewielkie i dysponują odpowiednią mocą obliczeniową potrzebną
do przetwarzania sygnałów wysyłanych na odpowiednie kanały. Przetwarzanie
sygnału na z cyfrowego na analogowy oraz jego wzmocnienie jest realizowane
przez moduł Amp Zero pHAT firmy JustBoom. Jest to połączenie wysokiej jakości
przetwornika analogowo-cyfrowego oraz wzmacniacza klasy D o mocy wyjściowej
$2\times40\si{\watt}$. Sygnał może być próbkowany z częstotliwością
\SI{192}{\kilo\hertz} i rozdzielczością \SI{32}{\bit}, z kolei zastosowanie
wzmacniacza klasy D umożliwia osiąganie odpowiedniej mocy głośników bez
zużywania nadmiernej ilości energii oraz zachowując niewielkie rozmiary
pojedynczych modułów. 
Wybrano głośniki firmy Faital Pro model 3FE25. Podstawowe parametry
przedstawiono w tabeli \ref{tab:paramglosnik}.

\begin{table}[!tbh]
  \centering
  \caption{Parametry głośnika Faital Pro 3FE25}
  \begin{tabular}{|c|c|} \hline
    \textbf{Parametr} & \textbf{Wartość} \\ \hline
    Średnica & \SI{3}{\inch} \\ \hline
    Impedancja nominalna & \SI{8}{\ohm} \\ \hline
    Moc nominalna & \SI{20}{\watt} \\ \hline
    Skuteczność (\SI{1}{\watt} / \SI{1}{\metre}) & \SI{91}{\decibel} \\ \hline
    Zakres częstotliwości & od \SI{100}{\hertz} do \SI{20}{\kilo\hertz} \\ \hline
  \end{tabular}
  \label{tab:paramglosnik}
\end{table}

Ponieważ moduły wzmacniaczy oraz przetworników A/C były dwukanałowe, na każdy
moduł oraz komputer Raspberry Pi przypadły dwa głośniki. W celu zwiększenia
możliwości adaptacji systemu do wymaganych warunków postanowiono każdy głośnik
umieścić w oddzielnej obudowie tak, aby istniała możliwość zmiany zarówno
odległości jak i kąta pomiędzy poszczególnymi żródłami macierzy głośnikowej.


\section{Wnioski}



\printbibliography

\end{document}
