% !TeX spellcheck = pl_PL
\documentclass[12pt]{oska}

% Lista wszystkich języków stanowiących języki pozycji bibliograficznych użytych w pracy.
% (Zgodnie z zasadami tworzenia bibliografii każda pozycja powinna zostać utworzona zgodnie z zasadami języka, w którym dana publikacja została napisana.)
\usepackage[english,polish]{babel}

% Użyj polskiego łamania wyrazów (zamiast domyślnego angielskiego).
\usepackage{polski}

\usepackage[utf8]{inputenc}

% dodatkowe pakiety
\usepackage{mathtools}
\usepackage{amsfonts}
\usepackage{amsmath}
\usepackage{amsthm}
\usepackage[dvipsnames]{xcolor}
\usepackage{textcomp}

% obrazki
\usepackage{graphicx}
\usepackage{rotating}
\usepackage{caption}
\usepackage{float}

% --- < bibliografia > ---

\usepackage[
style=numeric,
sorting=none,
% Zastosuj styl wpisu bibliograficznego właściwy językowi publikacji.
language=autobib,
autolang=other,
% Zapisuj datę dostępu do strony WWW w formacie RRRR-MM-DD
%urldate=iso,
%seconds=true,
% Nie dodawaj numerów stron, na których występuje cytowanie
backref=false,
% Podawaj ISBN.
isbn=true,
% Nie podawaj URL-i, o ile nie jest to konieczne
url=false,
% Ustawienia związane z polskimi normami dla bibliografii
maxbibnames=6,
minbibnames=6,
% Jeżeli używamy Bibera:
backend=biber
]{biblatex}

\AtBeginBibliography{
\renewcommand\labelnamepunct{:\space}
\renewcommand\newunitpunct{\addcomma\space}
\renewcommand{\finentrypunct}{}

\renewcommand{\bibopenparen}{\addcomma\addspace}
\renewcommand{\bibcloseparen}{\addspace}
}

\usepackage{csquotes}
% Ponieważ `csquotes` nie posiada polskiego stylu, można skorzystać z mocno zbliżonego stylu chorwackiego.
\DeclareQuoteAlias{croatian}{polish}

% Przecinki do numerów
\usepackage{icomma}
% ------------------------

% --- < listingi > ---

% Użyj czcionki kroju Times.
\usepackage{newtxtext}

\usepackage{listings}
\lstloadlanguages{TeX}

\lstset{
	literate={ą}{{\k{a}}}1
           {ć}{{\'c}}1
           {ę}{{\k{e}}}1
           {ó}{{\'o}}1
           {ń}{{\'n}}1
           {ł}{{\l{}}}1
           {ś}{{\'s}}1
           {ź}{{\'z}}1
           {ż}{{\.z}}1
           {Ą}{{\k{A}}}1
           {Ć}{{\'C}}1
           {Ę}{{\k{E}}}1
           {Ó}{{\'O}}1
           {Ń}{{\'N}}1
           {Ł}{{\L{}}}1
           {Ś}{{\'S}}1
           {Ź}{{\'Z}}1
           {Ż}{{\.Z}}1,
	basicstyle=\footnotesize\ttfamily,
}

% ------------------------

\AtBeginDocument{
	\renewcommand{\tablename}{\textbf{Tabela}}
	\renewcommand{\figurename}{\textbf{Rysunek}}
}

% ------------------------
% --- < tabele > ---

\usepackage{array}
\usepackage{tabularx}
\usepackage{multirow}
\usepackage{booktabs}
\usepackage{makecell}
\usepackage[flushleft]{threeparttable}

% defines the X column to use m (\parbox[c]) instead of p (`parbox[t]`)
\newcolumntype{C}[1]{>{\hsize=#1\hsize\centering\arraybackslash}X}

\setlength{\cftsecnumwidth}{10mm}
\setcounter{secnumdepth}{4}
\brokenpenalty=10000\relax


%---------------------------------------------------------------------------

\titlePL{Tytuł pracy w języku polskim}
\titleEN{Tytuł pracy w języku angielskim}
\affiliation{Afiliacja uczelni}

%------------------------------------AUTORZY-----------------------

\namem{Imię głównego autora}
\surnamem{Nazwisko głównego autora}
\email{Adres e-mail} % adres do korespondencji -- zazwyczaj głównego autora

% Jeśli jesteś jedynym autorem pracy - pozostaw poniższe pola puste

\namei{Imie współautora 1}
\surnamei{Nazwisko współautora 1}

\nameii{Imię współautora 2}
\surnameii{Nazwisko współautora 2}

\nameiii{Imię współautora 3}
\surnameiii{Nazwisko współautora 3}

\nameiiii{Imię współautora 4}
\surnameiiii{Nazwisko współautora 4}

\nameiiiii{Imię współautora 5}
\surnameiiiii{Nazwisko współautora 5}

%--------------------------STRESZCZENIE------------------------

\summaryPL{Tutaj wpisz streszczenie w języku polskim. Streszczenie powinno zawierać podstawowe tezy pracy, krótki opis metody badań i krótkie wnioski. Nie należy zamieszczać wzorów, przypisów literaturowych, obrazków itd. Sugerowana objętość streszczenia to 150 słów.}

\summaryEN{Tutaj wpisz streszczenie w języku angielskim -- wskazówki jak wyżej.}

%---------------------------------------------------------
% Nazwa pliku z bibliografią
%---------------------------------------------------------
\addbibresource{bibliografia.bib}


\begin{document}

\maketitles

\section{Wstęp}

Wprowadzenie w zagadnienie omawiane w dalszej części artykułu. Wstęp ani żadna część artykułu nie może zawierać zdań skopiowanych ze streszczenia. Język nie może zawierać kolokwializmów, skrótów myślowych itd. Należy unikać pojedynczych liter na końcu linijki -- w razie potrzeby łącząc literę z kolejnym wyrazem tyldą.

Każdy akapit powinien mieć wciętą pierwszą linijkę -- nie należy zmieniać podanego formatowania.

\section{Druga część}

Przypisy literaturowe w kolejności występowania w tekście~\cite{bib1}. Kropka kończy zdanie (nawiasy, odwołania itd. powinny być przed kropką).

Wzory matematyczne powinny być wyśrodkowane z numeracją w nawiasie okrągłym wyrównaną do prawej strony. Wzór stanowi część zdania, więc przed wzorem powinien być dwukropek, po wzorze powinna być kropka lub przecinek:

\begin{equation}
	T = \frac{0,161 V}{\alpha S} \quad, \label{wzor1}
\end{equation}


\noindent gdzie:\\
V -- objętość, \si{\meter\cubed}\\
$\alpha$ -- współczynnik pochłaniania dźwięku\\
S -- pole powierzchni ścian ograniczających, \si{\meter\squared}\\

Rysunki powinny być wyśrodkowane, do każdego rysunku powinno być odwołanie w~tekście pracy (Rysunek~\ref{rys1}). Podpis znajduje się pod rysunkiem.

\begin{figure}[H]
	\centering
	%\includegraphics[width=\textwidth]{sample_rys1.png}
	\caption{Jeśli rysunek nie jest wykonany przez autora pracy, w~podpisie powinien znaleźć się przypis~\cite{bib2}. Jeśli rysunek pochodzi ze strony internetowej, można podać adres strony. Wielkość czcionki w~opisie osi na rysunku należy dobrać tak, aby była czytelna.}
	\label{rys1}
\end{figure}

Tabelę opisuje się u góry, do każdej tabeli powinno być odwołanie w tekście (Tabela~\ref{tab1}).

\begin{table}[H]
	\centering
	\caption{Zestawienie średnich wyników pomiarów fizycznego współczynnika pochłaniania dźwięku wielowarstwowego kompozytu (według \cite{bib2}).}
	\label{tab1}
	\begin{tabular}{|c|c|c|c|c|c|c|c|}
		\hline
		\multirow{3}{*}{\textbf{Rodzaj materiału}} & \multicolumn{7}{|c|}{\textbf{Współczynnik pochłaniania dźwięku} $\boldsymbol{\alpha_f}$} \\\cline{2-8}
		& \multicolumn{6}{|c|}{\textbf{pasmo częstotliwości \textit{f},
	    \si{\hertz}}} & \textbf{średnia}\\\cline{2-8}
		& \textbf{125} & \textbf{250} & \textbf{500} & \textbf{1000} & \textbf{2000} & \textbf{4000} &\\\hline
		Materiał 1 & 0,07 & 0,26 & 0,26 & 0,17 & 0,10 & 0,07 & 0,16 \\\hline
		Materiał 2 & 0,02 & 0,12 & 0,24 & 0,36 & 0,45 & 0,70 & 0,32 \\\hline
	\end{tabular}
\end{table}

\section{Wnioski}

Wnioski powinny zawierać zwięzłe podsumowanie pracy i komentarz dotyczący uzyskanych wyników. W tej części można również opisać dalsze prace, jakie należy podjąć w celu dalszego rozwoju tematu. Nie należy kopiować zdań ze streszczenia czy innej części artykułu. Wszystkie pozycje w literaturze muszą być powołane w tekście~\cite{bib3},~\cite{bib4},~\cite{bib5}. 

\printbibliography

\end{document}
