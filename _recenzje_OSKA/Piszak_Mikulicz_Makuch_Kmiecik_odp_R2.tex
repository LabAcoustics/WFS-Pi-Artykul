% !TeX spellcheck = pl_PL
\documentclass[12pt]{article}
\usepackage[left=2.5cm, right=2.5cm]{geometry}

\usepackage[english,polish]{babel}
\usepackage{color}

% Użyj polskiego łamania wyrazów (zamiast domyślnego angielskiego).
\usepackage{polski}
\usepackage[utf8]{inputenc}
\pagestyle{empty}
\usepackage{indentfirst}

\usepackage{siunitx}

\parindent 10mm
\parskip 3mm

\begin{document}

    \begin{center}
        \textbf{Odpowiedź na recenzję }\\
        \vspace{10pt}
        Tytuł artykułu: \textbf{Wielokanałowy system syntezy pola akustycznego}
        \\
        Autorzy: \textbf{Marcel Piszak, Szymon Mikulicz, Teresa Makuch, Michał
        Kmiecik}
    \end{center}

Dziękujemy za recenzję pracy i~wskazanie elementów wymagających poprawy.
Poniżej odniesienie do poszczególnych punktów recenzji.

\section{Ogólny pogląd na treść, układ, oraz wartość
naukową artykułu}

Rozwinięto wstęp o~przegląd literatury?

\textit{Czy mamy dopisać wyniki pomiarów???}

\section{Ocena poprawności terminologii i użytych sformułowań}

Zastosowano się do zaproponowanych poprawek, rozwijając wskazane opisy
i~poprawiając poszczególne sformułowania.

% \subsection{Tytuł – WFS z założenia wykorzystuje „wielokanałowość”}
%
% Zmieniono tytuł pracy z "Wielokanałowy system syntezy pola akustycznego" na
"System syntezy pola akustycznego"
%
% \subsection{Streszczenie}
% \textcolor{red} {Wykminić o co chodzi}
%
% \subsection{rozkład pola akustycznego” (str.2.-1. wiersz od dołu oraz str.4.-
% 5. wiersz od dołu) – potoczne,nieprawidłowe sformułowanie w tym kontekście}
%
% Zmieniono na odpowiednio rozkład potencjału i ciśnienia akustycznego.
%
% \subsection{„3.1.Geometria” – konieczne doprecyzowanie}
%
% Zmieniono na ,,Wymiary macierzy''
%
% \subsection{6.„w oparciu o język Rust” (str.7.-9. wiersz od dołu) – „w
% języku”, „wykorzystuj c”?}
% Zmieniono na \dots wykorzystując język Rust \dots
%
% \subsection{7.„ilość przewodów” (str.7.-4. wiersz od dołu) – „liczbę”
% (policzalne)}
% Poprawiono.
%
% \subsection{8.„Zło ono … zło onym problemem” (str.1.- od 6. do 4. wiersza od
% dołu)}
% Zmieniono na: ,,Złożoność
% zjawisk związanych z propagacją fali akustycznej w rzeczywistym środowisku
% sprawia, że omawiane zagadnienie staje się skomplikowanym problemem.''
%
% \subsection{9.„Istnienie takiego … takich systemów…” (str.1.- 1. wiersz od
% dołu)}
% Istnienie takiego ograniczenia wyklucza zastosowanie wyżej opisanych systemów
% w~sytuacjach, gdy mamy do czynienia z~odbiorcą przemieszczającym się w strefie
% odsłuchu.
%

\section{Ocena celowości załączonego materiału ilustracyjnego}

Poprawiono podpis pod rysunkiem~1 oraz dodano rozwinięcie skrótu GPIO w~tekście
(piny ogólnego przeznaczenia, ang. \textit{General Purpose Input-Output}).

Na rysunku~2 nie podano jednostek, ponieważ dla lepszej czytelności wykresu
wysokie wartości ciśnienia tuż przy źródłach zostały zaokrąglone, a~następnie
dokonano normalizacji. Zabieg taki usprawiedliwia cel przedstawienia wykresów,
którym było ukazanie rozkładu ciśnienia w~polu dalekim, nie zaś przedstawienie
dokładnych wartości ciśnienia. Dodano odpowiedni komentarz wyjaśniający.

% \subsection{Rysunek 1.}
%
% \textcolor{red}{Rozkminić o co chodzi}
%
% \subsection{Rysunek 2.}
% Dodano [Pa] w komentarzu
%
% \subsection{Rysunek 3. – brak opisu zastosowanych skrótów, GPIO?}
%
% Dodano angielskie rozwinięcie skrótu GPIO w tekście oraz  objaśnienia
% w~podpisie pod rysunkiem 3.


\end{document}
