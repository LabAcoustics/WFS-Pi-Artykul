% !TeX spellcheck = pl_PL
\documentclass[12pt]{article}
\usepackage[left=2.5cm, right=2.5cm]{geometry}

\usepackage[english,polish]{babel}
\usepackage{color}

% Użyj polskiego łamania wyrazów (zamiast domyślnego angielskiego).
\usepackage{polski}
\usepackage[utf8]{inputenc}
\pagestyle{empty}
\usepackage{indentfirst}

\usepackage{siunitx}

% \usepackage{titlesec}
% \titleformat{\section}[display]{}{\thesection.}{}{}

\usepackage{sectsty}
\allsectionsfont{\normalsize}

\parindent 10mm
\parskip 3mm

\DeclareRobustCommand{\BibTeX}{%
  {\normalfont B\kern-.05em{\scshape i\kern-.025em b}\kern-.08em \TeX}%
}

\begin{document}

    \begin{center}
        \textbf{\large Odpowiedź na recenzję }\\
        \vspace{10pt}
        Tytuł artykułu: \textbf{Wielokanałowy system syntezy pola akustycznego} \\
        Autorzy: \textbf{Marcel Piszak, Szymon Mikulicz, Teresa Makuch, Michał Kmiecik}
    \end{center}

    Dziękujemy za szczegółową recenzję artykułu, zwrócenie uwagi na pewne
    niejasności oraz zbyt mało obszerne wyjaśnienia. Poniżej odniesienie do
    poszczególnych punktów recenzji.

    \section{Ogólny pogląd na treść, układ, oraz wartość naukową artykułu}

    \textcolor{red}{Dodatkowa analiza literatury?}

    \section{Ocena poprawności terminologii i~użytych sformułowań}

    Zastosowano większość sugerowanych zmian przedstawionych w~uwagach ogólnych. Poniżej uzasadnienie dla fragmentów, które pozostawiono bez zmian.

    \textit{Istnienie takiego ograniczenia wyklucza zastosowanie wyżej opisanych systemów w sytuacjach, gdy mamy do czynienia z odbiorcą przemieszczającym się w~strefie odsłuchu.}

    Zdaniem Autorów to zdanie wystarczająco jasno opisuje sytuację, o~jakiej jest mowa i~nie wymaga uzupełnienia o~przykład, który utrudniłby zrozumienie zdania.

    \textit{We wzorze 3 nie występuje całka, dlatego ten opis nie jest pełny.}

    Określenie całka Helmholtza-Huygensa odnosi się do teorii opisu pola akustycznego, z~której korzysta wzór~3, a~nie dotyczy go bezpośrednio (wzór zaczerpnięto z~pozycji [3] bibliografii).

    \section{Ocena celowości załączonego materiału ilustracyjnego}

    \textit{Rys. 2 nie jest do końca jasny\dots Ponadto nasuwa się pytanie\dots}

    W~dolnej części rysunku~2 przedstawiono wyniki uzyskane dla 16~głośników, czyli właśnie dla metody WFS, symulującej falę akustyczną od źródła przedstawionego w~górnej części rysunku.

    Dla zwiększenia czytelności podpisano każdy rysunek osobno i~poprawiono opisy w~tekście.

    \section{Pozostałe komentarze, uwagi}

    \textit{Bibliografia: Zapisy nie są jednolite i~niepełne, rok powinien być na końcu, zamiast pełnych imion --- inicjały.}

    Bibliografia została wykonana zgodnie z~przygotowanym przez organizatorów konferencji szablonem, który korzysta z~systemu bibliograficznego \BibTeX, według wiedzy autorów opartego o~powszechnie przyjęte standardy.


\end{document}
